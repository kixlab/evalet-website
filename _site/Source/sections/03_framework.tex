\section{Functional Fragmentation: An Evaluation Approach}

To evaluate the alignment of an LLM, practitioners must not only quantify quality through numeric scores but also qualitatively understand how this models' outputs are composed and their characteristics~\cite{ribeiro2020beyond, dunlap2024vibecheck, gero2024supporting}.
While existing LLM-based evaluation methods can assess outputs across various criteria~\cite{zheng2023judging, kim2024evallm, ye2023flask}, they only provide holistic judgments (i.e., overall scores, justification) for each dimension.
Thus, practitioners must manually inspect outputs to identify the specific elements that satisfy or violate their goals.

To address this, we introduce \approach{}, an LLM-based evaluation method that \emph{decomposes} model outputs into criterion-relevant \emph{fragments} and then infers each fragment’s \emph{function}---i.e., the role or effect it serves that influences the output's fulfillment of that criterion. 
Our approach draws inspiration from inductive coding~\cite{thomas2006general} (i.e., interpreting raw data into codes based on higher-level themes) and rubric design~\cite{perlman2003performance} (i.e., inspecting artifacts to define quality aspects to review).
In this section, we outline the novel affordances that are supported by this approach for \textbf{inspecting}, \textbf{rating}, and \textbf{comparison} of LLM outputs.

% ------------------------------------------

\subsection{Inspect}
\label{framework:inspect}

\paragraph{Fragment-Level}
To make sense of existing LLM-based evaluations, practitioners must manually review outputs, connect them with the evaluator's justifications, and interpret each fragment's significance in terms of the criteria~\cite{kim2024evallm}.
Our approach automatically extracts criteria-relevant fragments and interprets their functions with respect to each criterion---directly presenting practitioners with qualitative interpretations to inspect and verify.
As one fragment can serve different functions under different criteria, our approach allows practitioners to examine the same content from multiple perspectives and identify trade-offs.
Given that criteria are often subjective, our approach can also uncover meaningful functions that the practitioner may have not previously considered---similar to \textit{inductive coding}~\cite{thomas2006general}.
Conversely, if the LLM evaluator extracts functions that the practitioner considers irrelevant to the criterion, practitioners can directly flag these to be ignored in future evaluations.

\paragraph{Output-Level}
Beyond identifying each fragment-level function in an output, practitioners may also need to inspect how these functions appear together in the output.
For example, when evaluating \criterion{Tension} in LLM-generated horror stories, a practitioner may need to understand how the LLM uses various functions to gradually build tension in the story.
Traditionally, this would require the practitioner to read the whole story, but not all of the content may be directly related to that criterion.
With \approach{}, each output can be summarized into a list of functions related to a given criterion, allowing practitioners to easily inspect each output by focusing only on the aspects of interest.

% ------------------------------------------

\subsection{Rate}
\label{framework:rate}

\paragraph{Fragment-Level}
By disentangling outputs into fragment-level functions, each individual function's alignment with a criterion can be rated independently. 
More fine-grained evaluations can support interpretability by clearly highlighting the specific aspects of an output that are aligned or misaligned with a criterion.
Instead of correcting the LLM evaluator by editing criteria descriptions, practitioners can directly re-rate specific functions to control future evaluations---similar to how educators develop rubrics by assessing examples of student work~\cite{perlman2003performance}. 
Beyond supporting practitioners, LLM evaluators are also more consistent when performing more fine-grained evaluations through checklists~\cite{saad2024lmunit, lin2024wildbench} or rubrics~\cite{kim2023prometheus, ye2023flask, kim2024biggen}.
Unlike these approaches, which rely on predefining these checklists and rubrics, our fragment-level functions are \textit{emergent}, identified dynamically based on the outputs and criteria.

\paragraph{Output-Level}
Instead of providing uninterpretable and opaque scores (e.g., 2 out of 5) for each model output, our approach enables us to rate each output based on its proportion of aligned and misaligned fragment-level functions (e.g., 75\% of surfaced functions are aligned)\footnote{For simplicity, we opt for equal weighting of each function. As discussed in Limitations, future work can explore automatic or manual approaches for weighting the significance of each function.}.
This provides a more interpretable signal of \textit{how much} misalignment there is in an output and why---allowing practitioners to understand what are the specific errors that need to be corrected~\cite{ribeiro2020beyond}.

% ------------------------------------------

\subsection{Compare}
\label{framework:compare}

\paragraph{Fragment-Level}
Comparing fragments across outputs can reveal common model behaviors, but directly comparing raw text is difficult because fragments may differ lexically or semantically even when they serve the similar function.
For example, when we evaluate an essay-writing LLM on \criterion{Logical Coherence}, these two sentences serve the same function as cohesive devices for a conclusion despite their wording differences: \textit{``In conclusion, the trend is clear.''}, and \textit{``To sum up, it supports our view.''}
By labeling each fragment's functions, our approach allows for comparison and grouping of fragments not based on their lexical similarity, but by their functional similarity---allowing practitioners to distill high-level insights and patterns.

\paragraph{Output-Level}
By considering each output as a list of its fragment-level functions, we can also compare outputs based on whether they share a function or set of functions.
For example, practitioners could group and filter outputs based on the inclusion of a specific function of interest and even calculate the distribution of outputs that contain certain function patterns---supporting the common practice of slicing data into subsets of interest in ML evaluation~\cite{cabrera2023zeno, wu2019errudite, sivaraman2025divisi}.
Beyond comparing outputs from a single LLM, practitioners could qualitatively compare the behaviors of different LLMs by comparing the distributions of specific functions in each model's outputs.