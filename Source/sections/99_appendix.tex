\appendix

\begin{figure}
    \centering
    \includegraphics[width=1.0\columnwidth]{figures/system_filter}
    \caption{In the Database Tab, users can browse through the list of base clusters for fragment-level functions from each output. By clicking on a cluster, users can view all outputs that contain any functions that are included in the selected cluster. Additionally, \sysname{} provides statistics summarizing the evaluation results for these outputs and clusters that contain functions that co-occur frequently with functions in the selected cluster.}
    \label{fig:system_filter}
    \Description{This figure shows how users can inspect outputs that include selected clusters in the Database Tab. The top section presents the selected clusters with summary statistics, such as the number of matching outputs and average scores. Below, commonly co-occurring clusters are shown for additional context. The bottom section displays outputs with its input, model response, and function-level evaluations, highlighting how the selected clusters appear within the output.}
\end{figure}

\section{LLM Prompts}
\label{appendix:prompts}

Here, we present the various LLM prompts that power \sysname{}: functional fragmentation and evaluation prompt (Fig.~\ref{fig:evaluation_prompt1}, ~\ref{fig:evaluation_prompt2}), base cluster creation prompt (Fig.~\ref{fig:base_cluster_prompt}), super cluster creation prompt (Fig.~\ref{fig:super_cluster_prompt}), super cluster deduplication prompt (Fig.~\ref{fig:super_cluster_deduplication_prompt}), and prompt to reassign base clusters to super clusters (Fig.~\ref{fig:super_cluster_reassignment_prompt}). 

\section{Technical Evaluation Details}

In our technical evaluation, we compared our functional fragmentation approach against a baseline that provided evaluations at the output-level. We compared the two approaches in two tasks: fragment extraction and overall assessment.

\subsection{Fragment Extraction}
\label{appendix:tech_eval_extract}

\subsubsection{Dataset}

For the technical evaluation of fragment extraction, we used the Scarecrow dataset~\cite{wu2023fine}, which contains LLM-generated passages where human annotators annotated errors according to given categories.
As each data point in the dataset includes annotations from 10 annotators with varying granularities (e.g., word, phrase, sentence), we aggregate the annotations by selecting sentences where the majority of annotators agreed on a specific error type.
Then, we filter the data to only points with at least 3 annotations that were agreed on by the majority of annotators, which yielded 402 data points.

In this task, each data point could include different types of errors. For each data point, we assessed it only on the criteria that were related to the errors that were actually annotated for that data point. Specifically, we used the following criteria to encompass the error types in the dataset:
\begin{itemize}
    \item \textbf{\criterion{Language Quality}}: \textit{"This criterion captures a broad range of textual problems that degrade the clarity, correctness, or relevance of a passage. Issues with 'Language Quality' may include incorrect or awkward grammar and usage (e.g., missing, extra, or out-of-order words), irrelevance or contradiction with the given prompt ('off-prompt'), excessive or repetitive phrasing ('redundant'), internally conflicting statements ('self-contradiction'), or any general lack of clarity rendering the text confusing ('incoherent'). Any error that hinders readers’ understanding or undermines the text’s fidelity to the prompt falls under this umbrella."} (Covers the error types: "Grammar and Usage", "Off-prompt", "Redundant", "Self-Contradiction", and "Incoherent")
    \item \textbf{\criterion{Factual Accuracy}}: \textit{"This criterion encompasses all errors that compromise the factual correctness of a passage. Issues with 'Factual Accuracy' include mathematical or numerical mistakes ('bad math'), incorrect factual assertions contrary to well-known information ('encyclopedic' errors), and any statements that violate fundamental common sense ('commonsense' errors). Any generation that misrepresents or distorts verifiable information, basic knowledge, or logical reasoning falls into this category."} (Covers the error types: "Bad Math, "Commonsense", and "Encyclopedic")
    \item \textbf{\criterion{Reader Accessibility}}: \textit{"This criterion covers situations where the content demands more effort than usual for the average reader to comprehend or verify. Issues with 'Reader Accessibility' may arise if the text includes claims that require external verification ("needs Google") or relies on technical or domain-specific vocabulary beyond common knowledge ('technical jargon'). While these issues do not necessarily render the text incorrect, they make the content harder to assess or understand without additional expertise or resources."} (Covers the error types: "Needs Google" and "Technical Jargon")
\end{itemize}

\subsubsection{Measures}

We compute the Intersection-over-Union (IoU) between extracted fragments and the ground-truth annotations.
For each error type or criterion, we calculate the number of tokens shared by both the extracted fragments and the ground-truth annotations (i.e., intersection) and divide that by the number of tokens that appear in either set (i.e., union).
We also evaluate extraction performance using precision, recall, and F1-score at the sentence level. 
For each approach, we identify all sentences containing extracted fragments and all sentences containing ground-truth fragments, and then count matches between these sentences as correct predictions.
We opted for sentence-level matching due to the granularity differences between the fragments from each approach and the ground-truth annotations.

\subsection{Overall Assessment}
\label{appendix:tech_eval_overall}

\subsubsection{Dataset}

We use the RewardBench dataset~\cite{lambert2024rewardbench}, which contains input prompts and two responses generated by different LLMs, where one response was \textit{chosen} (i.e., preferred by a majority of human annotators) and the other was \textit{rejected}.
The dataset is a collection of multiple different datasets and the data points are assigned to different subsets depending on their category: Chat, Chat Hard, Safety, and Reasoning.
In our experiments, we exclude the Reasoning subset as it encompasses almost as much data as all of the subsets combined, but focuses solely on math and coding-related prompts.
As our method focuses on the evaluation of long-form text with multiple text fragments, we filtered the dataset to only cases where both responses were at least 100 words in length (i.e., one paragraph or longer)---yielding 432 data points.

For the overall assessment task, we used \textbf{\criterion{Human Preference}} as the criterion: \textit{"Does the response align closely with human judgment and preferences, reflecting the naturalness, usefulness, and appropriateness that will be valued by the user? This includes considering user satisfaction, appropriateness of tone, style, and context- specific nuances that resonate positively with human evaluators."}

\subsubsection{Measures}

We used each approach to independently evaluate each response in a pair and then compared the evaluation scores for each response to determine the predicted \textit{chosen} response.
Specifically, for \texttt{Ours}, the score for each response was the ratio of positively rated functions out of all extracted functions. 
For \texttt{Rating}, we used the rating (1 to 5) given to each response.
Then, we calculated the \textit{accuracy} of each approach in correctly determining the \textit{chosen response}---where ties are considered as incorrect.


\section{Study Datasets}
\label{appendix:study_datasets}

\paragraph{Task Dataset Construction Process} In our study, participants explored the outputs and evaluations for two different long-form generation tasks: (1) short horror story generation, and (2) social media advertisement post generation.
For each task, we created an initial dataset of 100 inputs: (1) three keywords for the horror story task (e.g., "closet, eyes, sigh"), and (2) short phrases that describe a product for the advertisement task (e.g., "posture correcting smart backpack").
To create these datasets, we started with 5 manually crafted examples. Then, given these examples, we used \texttt{gpt-4o-2024-11-20} to gradually synthesize more data in steps: generate 10 more data points, manually verify these data points, filter out low-quality ones, and then repeat by using all of the created data as examples.

\paragraph{Task Criteria}
These tasks were evaluated in the following criteria (translated from Korean):
\begin{itemize}
    \item Horror Stories - \textbf{\criterion{Horror Atmosphere}}: \textit{"This criterion assesses how effectively the story creates immersive and constant fear or psychological anxiety. This criterion should evaluate a story positively if it: (1) creates fear through implicit suggestions instead of explicit explanations; (2) includes "Aha!" moments that lead readers to reconsider previous occurrences in the story; or (3) reveals new scary elements when re-reading the story. A story is evaluated negatively if: (1) the story relies on traditional or cliché elements; (2) scary elements are not implied but instead explicitly explained; or (3) there are no moments that turn the story into a scary or fearful mood."}
    \item Advertisements - \textbf{\criterion{Emotional Effect}}: \textit{"This criterion assesses how effectively the advertisement elicits a meaningful and authentic emotional response from readers. In particular, it focuses on whether the advertisement can naturally stimulate emotions within the hearts of consumers (e.g., joy, nostalgia, inspiration, empathy, excitement, and warmth). The advertisement should not convey emotions in an artificial or forced way, and should elicit empathy without exaggerations. Good advertisements should naturally connect emotional experiences with brands or products to increase consumer trust, strengthen connections, and make a strong enough impression that prompts the consumer to act on these feelings."}
\end{itemize}

During the study, participants were asked to select a new criterion for one of the tasks and to run new evaluations. These were the list of criteria that were provided to participants for each task:

\begin{itemize}
    \item Horror Stories
    \begin{itemize}
        \item \textbf{\criterion{Psychological Depth}}: \textit{"This criterion evaluates how realistically the story portrays the internal psychology and emotions of characters. This criterion should evaluate a story positively if: (1) characters' emotions are convincingly depicted, or (2) readers can empathize with characters' inner conflict and anxiety. A story should be evaluated negatively if: (1) characters' emotions are superficial or simplistic, or (2) characters' responses are unrealistic or contrived."}
        \item \textbf{\criterion{Creative Originality}}: \textit{"This criterion evaluates how the story presents horror elements in a fresh and unique manner. This criterion should evaluate a story positively if: (1) presents common horror elements but with unexpected perspectives or situations, or (2) the source of horror avoids common clichés and is realized through unique ideas. A story is evaluated negatively if: (1) relies on common clichés, or (2) directly replicate approaches commonly used in prior famous stories."}
        \item \textbf{\criterion{Keyword Integration}}: \textit{"This criterion evaluates how naturally and creatively the keywords are integrated in the story. This criterion should evaluate a story positively if: (1) keywords naturally link with the horror atmosphere, (2) keywords play a crucial role in generating horror, or (3) keywords provide significant clues that help readers uncover hidden meanings or plot twists. A story is evaluated negatively if: (1) keywords feel artificially inserted and are unrelated to the story's flow, (2) keywords do not meaningfully contribute to horror or plot progression, or (3) removing keywords would not significantly impact the horror of the story."}
    \end{itemize}
    \item Advertisements
    \begin{itemize}
        \item \textbf{\criterion{Creativity and Originality}}: \textit{"This criterion evaluates how effectively an advertisement captures reader's attention through creative and unique ideas. Advertisements must avoid mundane or predictable content, leaving a lasting impression through fresh perspectives or innovative expressions. The advertisement should distinguish itself from existing ads through memorable elements (e.g., original concepts, creative storytelling, or unexpected components)."}
        \item \textbf{\criterion{Brand Consistency and Message Clarity}}: \textit{"This criterion assesses how consistently the ad reflects a brand or image. Ads must be consistent in the tone, style, and content---conveying a clear and understandable message. The ad should focus on clear key aspects to allow consumers to effortlessly associate the ad with a brand without causing confusion or misunderstandings."}
        \item \textbf{\criterion{Call-to-Action Effectiveness}}: \textit{"This criterion evaluates how effectively the advertisement persuades consumers to actively engage with the product. Effective ads should not only attract attention or generate interest, but also lead to specific actions like purchases, website visits, product usage, or social media shares. Calls-to-action should organically motivate consumer behavior, providing incentives that are attractive and appear easily obtainable."}
    \end{itemize}
\end{itemize}

\paragraph{Pre-Identified Evaluation Issues for Each Task}
\label{appendix:study_evaluation_issues}
In the user study, for each task, participants were asked to correct the LLM evaluations based on two pre-identified issues with the evaluation results.
The issues were provided to participants in Korean during the study.
\begin{itemize}
    \item Horror Stories
    \begin{itemize}
        \item \textbf{Positive to Negative} - The LLM evaluator is currently providing positive evaluations to outputs that contain phrases that explicitly describe the fear experienced by the protagonist or a character. These case should be evaluated as negative.
        \item \textbf{Excluded} - The LLM evaluator is currently evaluating phrases that are ambiguous or vague for the "Horror Atmosphere" criterion (e.g., \textit{"something was watching me from the darkness"}). These should be assessed by a different criterion so they should be excluded.
    \end{itemize}
    \item Advertisements
    \begin{itemize}
        \item \textbf{Negative to Positive} - The LLM evaluator is currently providing negative evaluations to outputs that contain phrases that encourage a certain emotion or behavior from the consumer. These cases should be evaluated as positive.
        \item \textbf{Excluded} - The LLM evaluator is currently evaluating phrases that emphasize eco-friendliness for the "Emotional Effect" criterion (e.g., \textit{"your small choices can save the Earth"}. These should be assessed by a different criterion so they should be excluded.
    \end{itemize}
\end{itemize}

\section{Study Metrics}
\label{appendix:study_metrics}

\paragraph{Survey Questions}
In our post-task survey, participants were asked to rate their agreement with the following statements on a 7-point Likert scale (1 - "Strongly Disagree", 7 - "Strongly Agree"):
\begin{itemize}
    \item "I was able to identify critical or important issues with the model outputs."
    \item "I was able to identify critical or important issues with the model evaluations."
    \item "I am confident that I identified most issues with the model outputs."
    \item "I am confident that I identified most issues with the model evaluations."
    \item "I am confident that I can act on and resolve the issues that I identified with the model outputs."
    \item "I am confident that I can act on and resolve the issues that I identified with the model evaluations."
\end{itemize}

\paragraph{Calculating Success Rate for Correcting Evaluation Issues}
In the user study, participants refined evaluation criteria by adding few-shot examples and modifying descriptions to address the given LLM evaluation issues (Appendix~\ref{appendix:study_evaluation_issues}). 
To verify the effectiveness of these refinements, we created a test set containing outputs known to exhibit these issues under the original criteria.
Using the original method to create the datasets for our study, we generated an additional 100 data points per task, which were evaluated using the original criteria.
An author reviewed these evaluations to identify two common issues per task and selected five outputs per issue that demonstrated that issue. 
Specifically, each output contained an \textit{issue fragment}---i.e., a fragment that required an opposite rating (i.e., "positive to negative" or "negative to positive" issue type) or should have not been extracted (i.e., "excluded" issue type).

Participants' refined criteria were then applied to re-evaluate these test outputs, enabling us to measure their success rate in correcting evaluation issues.
We used the original evaluation prompt and evaluator LLM.
Due to evaluation inconsistencies even at temperature 0, each output was evaluated three times per participant.
We first identified if any of the newly extracted fragments matched the issue fragments by calculating their token-level IoU and considering a match if IoU was greater than 0.5.
Then, the evaluation issue was considered corrected as follows depending on the issue type: (1) \textbf{\textit{"Positive to Negative"} or \textit{"Negative to Positive"}} - Issue was corrected if there is a matching fragment and it received a rating opposite to the original rating; or (2) \textbf{\textit{Excluded}} - Issue was corrected if no matching fragment was extracted.
Each participant's success rate was computed per task based on the percentage of corrected cases out of the 30 evaluations (2 issues $\times$ 5 outputs per issue $\times$ 3 evaluations each).

\section{Example Cases Details}
\label{appendix:example_cases}

In our example cases (Sec.~\ref{sec:example_cases}), we test our \approach{} approach on (1) reasoning traces, (2) user-LLM conversations, (3) social agent simulations and (4) trajectories of computer use agents. Here, we provide additional details on the data used.

\subsection{Reasoning Traces}

We collect reasoning traces from the OpenThoughts dataset~\cite{openthoughts}, which contains the reasoning traces generated by DeepSeek-R1~\cite{guo2025deepseek} on various problems.
Specifically, the datasets consists of 6 task domains and we sampled 35 data points per domain, yielding a total of 210 data points.
We evaluated these on \criterion{Metacognitive Insight}: \textit{"Metacognitive Insight evaluates the degree to which the assistant actively reflects upon, regulate, and articulate their thought processes during reasoning. It assesses its capability to recognize assumptions, identify potential errors, reconsider decisions, adaptively guide their cognitive strategies, and more metacognitive abilities. High levels of Metacognitive Insight indicate advanced self-awareness and a proactive approach to managing its own cognitive activities."}

\subsection{User-LLM Conversations}

We applied our approach on the HarmfulQA dataset~\cite{bhardwaj2023redteaming}, which consists of questions designed to elicit harmful responses from LLMs (i.e., red teaming), along with corresponding harmless and harmful model responses.
We randomly sampled 200 data points and evaluated them for \criterion{Harmlessness}: \textit{"Harmlessness assesses the extent to which an assistant engages with users safely and respectfully during interactions. It evaluates whether the assistant’s responses consistently avoid causing harm or negative consequences, including emotional distress, misunderstandings, biases, offensive or inappropriate content, misinformation, and more. This criterion assesses the assistant’s overall ability to engage positively and respectfully, maintaining user trust and well-being throughout the interaction."}

\subsection{Social Agents}
We applied our approach on a dataset generated through the SOTOPIA environment~\cite{wang2024sotopia}. Each data point includes a dialogue that simulates negotiations between two LLM agents that are role-playing as characters with different social goals.
We randomly sampled 200 data points from the dataset and evaluated each dialogue based on \criterion{Social Intelligence}: \textit{"Social Intelligence evaluates an AI assistant's ability to effectively understand, navigate, and manage social interactions with other users or agents. It assesses how well the assistant interprets social contexts, emotional signals, conversational nuances, and interpersonal dynamics. A socially intelligent assistant demonstrates empathy, adaptability, emotional sensitivity, and the capacity to respond appropriately and naturally, enhancing the overall quality and realism of interactions."}

\begin{figure}
    \centering
    \includegraphics[width=1.0\columnwidth]{figures/case_study_appendix.pdf}
    \caption{Fragment-level functions and clusters identified from computer use agent trajectories under the Action Efficiency criterion. The visualization disentangles optimization strategies from behavioral inefficiencies.}
    \label{fig:case_study_appendix}
    \Description{A 2D scatter plot visualization showing clusters of agent behaviors related to Action Efficiency. The map reveals distinct groups of positive and negative behaviors. On the positive side, clusters represent optimization strategies, labeled with annotations such as "Optimization through Direct Manipulation" and "Keyboard Shortcuts for Efficiency." On the negative side, clusters highlight inefficiencies, featuring labels like "Redundant Navigation Issues" and "Excessive detail in simple explanation".}
\end{figure}

\subsection{Computer Use Agents}
\label{appendix:example_agents}
We applied our approach on the AgentNet dataset~\cite{wang2025opencua}, which contains the trajectory data of computer use agents including the agent's reasoning and actions.
We randomly sampled 200 data points from the dataset and concatenated thoughts and actions in each data point.
Then, we evaluated each agent's trajectory based on \criterion{Action Efficiency}: \textit{"This criterion evaluates the efficiency and economy of the action sequences executed by the agent to achieve a goal. It prioritizes the presence or absence of unnecessary actions over the logic of the underlying thought process."}

Figure ~\ref{fig:case_study_appendix} visualizes the landscape of fragment-level functions surfaced from the agents' traces, showing distinct behavioral patterns that binary success metrics often obscure.
On the positive side, the approach surfaces distinct optimization strategies that can distinguish between agents demonstrating expert-level proficiency via \textit{``Keyboard Shortcuts''} and those minimizing steps through \textit{``Direct Manipulation''} of the interface.
Conversely, our approach also surfaces distinct behavioral patterns that can lead to overall inefficient workflows from the agents, such as redundant actions (e.g., \textit{``Redundant Navigation Issues''}) and superfluous action instructions (e.g., \textit{``Excessive Detail in Simple Explanation''}).
The granularity afforded by the fragment-level functions allows practitioners to not only identify the agents' strengths, but also diagnose the root causes of their inefficiencies, facilitating targeted refinement.

\begin{figure*}
\begin{framed}
\begin{flushleft}
\noindent
\textbf{Functional Fragmentation and Evaluation (1)} \\

\medskip
\hrule
\bigskip

\textbf{System Prompt}

\begin{Verbatim}[breaklines, fontsize=\fontsize{5}{6}\selectfont]
You are a **meticulous, insightful, and critical evaluator**. Your primary role is to **evaluate AI assistant responses** based on specified **evaluation criteria**, carefully identifying **abstract features** that affect response quality.

You will receive:

1. **User's Instruction**: The prompt provided to the AI assistant.
2. **AI Assistant’s Response**: The assistant’s output based on the instruction.
3. **Evaluation Criteria**: Standards used to assess response quality.
- **Examples**: Each criterion includes the examples that should be evaluated as *positive*, *negative*, or should be *excluded* (i.e., out of scope for this criterion).

## Evaluation Steps

Evaluate the AI assistant's response individually against each given criterion. Conduct the steps for each criterion, focusing exclusively on one criterion at a time.

### Step 1: Thoroughly Familiarize Yourself with the AI Response

Carefully read the AI assistant’s response from start to finish to ensure you do not overlook any details. Confirm that you fully grasp its main ideas, structure, key points, and nuances. You should think aloud as you read, noting any impressions or observations that arise throughout the reading.

### Step 2: Extract All Relevant Fragments

Identify and extract **all fragments** (phrases, sentences, paragraphs, etc.) from the response that are directly relevant to the current evaluation criterion. If the entire response was relevant, you should return the token "$WHOLE$" instead of extracting the whole response. Your list of extracted fragments should be:
- **Exhaustive**: Include all relevant fragments that contribute to the evaluation of the criterion.
- **Balanced**: Ensure that you consider both positive and negative instances.

You should not include any fragments that are similar to the "Examples to Exclude" for each criterion. If the fragment should be excluded, you MUST mark the fragment as 'is_excluded' in your final response.

### Step 3: Analysis of Fragments

You should then analyze each fragment that you extracted based on its relationship to the criterion. For each fragment, analyze its:
- **Relevance**: How relevant is the fragment to the criterion?
- **Impact**: What impact does the fragment have on the overall response quality?
- **Implications**: What are the implications of this fragment on the criterion being evaluated?

### Step 4: Abstract Fragments into Features

Identify and abstract **specific features** present in the fragments that contribute to the criterion being evaluated. These features should be **abstract and generalizable** characteristics that can be applied to other responses. Each feature should be:
- **Distinct**: Clearly differentiable from other features.
- **Generalizable**: Applicable to a broader set of responses.
- **Abstract**: Not tied to specific details of the response.
- **Interpretable**: Clearly understandable and interpretable by others.
- **Concise**: Clearly and succinctly described.

#### Feature Examples:

- **Example 1:**
  - Criterion Name: "Engagingness"
  - Fragment: "Antibodies are like mini-soldiers that shoot down germs in your body to keep you healthy."
  - Feature: "Explaining concepts through metaphors"
- **Example 2:**
  - Criterion Name: "Child Safety"
  - Fragment: "Antibodies are like mini-soldiers that shoot down germs in your body to keep you healthy."
  - Feature: "References to mildly violent or harmful actions"
- **Example 3:**
  - Criterion Name: "Directness"
  - Fragment: "While your skills with ReactJS, Vue, and Svelte are impressive, we are unsure whether you may be a good fit for our company's architecture."
  - Feature: "Hedged communication with ambiguous decision outcome"
- **Example 4:**
  - Criterion Name: "Accessibility"
  - Fragment: "Start with a 5-day split: Monday—deadlifts (5x5 at 80% of 1RM), Wednesday—squats (4x6 at 75% of 1RM), Friday—bench press (5x5 at 80% of 1RM). Track progress weekly using linear periodization."
  - Feature: "Specialized instructions with technical jargon"
- **Example 5**
  - Criterion Name: "Inclusivity"
  - Fragment: "We must actively integrate diverse perspectives from historically excluded groups through initiative such as indigenous knowledge systems and community gatherings."
  - Feature: "Active suggestion for integrating diverse perspectives"

### Step 5: Feature Rating

Rate each feature’s alignment as:
- **Positive**: Meets or supports the criterion.
- **Negative**: Detracts from or misaligns with the criterion.

Consider the "Positive Examples" and "Negative Examples" when you judge whether each feature is positive or negative. These examples are provided by the user and they are IMPORTANT for your evaluation.
\end{Verbatim}

\bigskip

\end{flushleft}
\end{framed}
\caption{Prompt to fragment and evaluate functions from an output. (1/2)} 
\label{fig:evaluation_prompt1}
\end{figure*}

\begin{figure*}
\begin{framed}
\begin{flushleft}
\noindent
\textbf{Functional Fragmentation and Evaluation (2)} \\

\medskip
\hrule
\bigskip

\textbf{System Prompt}

\begin{Verbatim}[breaklines, fontsize=\fontsize{5}{6}\selectfont]
### Step 6: Provide an Overall Justification

Summarize your analyses for all fragments and features, providing a coherent and concise overall description of your evaluation. Avoid introducing new points in this section; instead, focus on summarizing the key points from your analyses.

You should then provide a short phrase that captures the essence of your justification. This phrase should be **concise and memorable**, encapsulating the main reasons behind your evaluation.

## Required YAML Output Format

Follow this exact YAML format precisely.

```yaml
evaluations:
  - criterion_name: <criterion>
    reading: |
      <thoughts and impressions as you read the response>
    fragments:
      - id: 1
        fragment: |
          <verbatim extracted fragment>
      - id: 2
        fragment: |
          <verbatim extracted fragment>
      # Additional fragments follow same structure, ensuring exhaustive and balanced coverage
    features:
      - fragment_id: 1
        analysis: |
          <analysis of the fragment>
        feature: |
          <abstract feature>
        is_excluded: <true/false, whether this fragment should be excluded according to the 'Examples to Exclude' for the criterion>
        alignment: <"positive"|"negative">
      - fragment_id: 2
        analysis: |
          <analysis of the fragment>
        feature: |
          <abstract feature>
        is_excluded: <true/false, whether this fragment should be excluded according to the 'Examples to Exclude' for the criterion>
        alignment: <"positive"|"negative">
      # Additional fragments follow same structure, ensuring exhaustive and balanced coverage
    overall_justification: <summarize the analyses for all fragments>
    keyphrase: <short phrase capturing the essence of your justification>

  - criterion_name: <next criterion>
    reading: |
      <thoughts and impressions as you read the response>
    fragments:
      - id: 1
        fragment: |
          <verbatim extracted fragment>
      # Additional fragments follow same structure, ensuring exhaustive and balanced coverage
    features:
      - fragment_id: 1
        analysis: |
          <analysis of the fragment>
        feature: |
          <abstract feature>
        is_excluded: <true/false, whether this fragment should be excluded according to the 'Examples to Exclude' for the criterion>
        alignment: <"positive"|"negative">
      # Additional features follow same structure
    overall_justification: <summarize the analyses for all fragments>
    keyphrase: <short phrase capturing the essence of your justification>
```

### YAML Formatting Guidelines
- Use exactly **2 spaces per indentation level**.
- Indent multiline texts (**analysis**, **fragment**, **feature**) by exactly **8 spaces**.
- Always use **|** to denote multiline texts.
- Avoid unnecessary blank lines or spaces.
\end{Verbatim}

\hrule
\bigskip

\textbf{User Prompt}

\begin{Verbatim}[breaklines, fontsize=\fontsize{5}{6}\selectfont]
### {criterion name}
  
**Description**: {criterion description}

**Positive Examples**
{positive examples}

**Negative Examples**
{negative examples}

**Excluded Examples**
{excluded examples}
\end{Verbatim}

\end{flushleft}
\end{framed}
\caption{Prompt to fragment and evaluate functions from an output. (2/2)} 
\label{fig:evaluation_prompt2}
\end{figure*}



\begin{figure*}
\begin{framed}
\begin{flushleft}
\noindent
\textbf{Create Base Clusters} \\

\medskip
\hrule
\bigskip

\textbf{System Prompt}

\begin{Verbatim}[breaklines, fontsize=\fontsize{5}{6}\selectfont]
You are tasked with summarizing a group of related statements into a short precise and accurate description and name.
Your goal is to create a concise summary that captures the essence of these statements and distinguishes them from other similar groups of statements.

## Context
The user will provide multiple sentences, where each sentence is a fragment from an LLM's generated response. Each fragment was selected by an evaluator because it is related to a specific evaluation criterion.
User want to gain insights from the cluster in the perspective of the criterion.

## Instruction
Summarize all the statements into a clear, precise, one-sentence description.
Your summary should reflect why these sentences are related to the criterion.
Your summary should be specific to this group and distinguish it from the contrastive answers of the other groups.

After creating the summary, generate a short name for the cluster of statements. This name should be at most ten words long (perhaps less) and be specific but also reflective of most of the statements.
The name should distinguish this group from the contrastive examples.
The name and summary should be written in the same language as the given statements or sentences.

## Warning
Do not elaborate beyond what you say in the tags. Remember to analyze both the statements and the contrastive statements carefully to ensure your summary and name accurately represent the specific group while distinguishing it from others.

## Response Format (in JSON)
```json
{
  "summary": <clear, precise, one sentence description about the group of sentence>,
  "name": <name at most ten words (or less) to represent the group of sentence>
}
```
\end{Verbatim}

\hrule
\bigskip

\textbf{User Prompt}

\begin{Verbatim}[breaklines, fontsize=\fontsize{5}{6}\selectfont]
### Sentences

- {sentences in the group}
\end{Verbatim}

\end{flushleft}
\end{framed}
\caption{Prompt to create base clusters from groups of functions.} 
\label{fig:base_cluster_prompt}
\end{figure*}


\begin{figure*}
\begin{framed}
\begin{flushleft}
\noindent
\textbf{Create Super Clusters} \\

\medskip
\hrule
\bigskip

\textbf{System Prompt}

\begin{Verbatim}[breaklines, fontsize=\fontsize{5}{6}\selectfont]
You are tasked with creating higher level cluster names based on a given list of clusters and their descriptions.
Your goal is to come up with broader categories that could encompass the concepts from lower level clusters.
  
## Context
The user will provide you with a list of clusters that encapsulate a group of related statement or information. You should analyze the themes and patterns in the clusters to create higher level cluster names that can group and represent the lower level clusters.
  
## Instruction
Your task is to create higher level cluster name that could potentially include all of the provided clusters.
If there are many clusters with a specific theme, ensure that the higher level cluster name retains sufficient specificity to illustrate the theme.
You should output one specific cluster name that can fully represent the provided clusters.
  
  1. Analyze the themes, topics, or characteristics common to multiple provided clusters.
  2. Create a name that is specific enough to be meaningful, but not so specific that it cannot meaningfully represent many different clusters.
  3. Ensure that the higher level cluster names are distinct from one another.
  4. Use clear, concise, and descriptive language for the cluster name.
  5. Use the same language as the original clusters for the new cluster names and descriptions.
  6. Provide concise description for each cluster in one sentence.
  
## Response Format (in JSON)
```json
{
  "description": <concise description for the higher level cluster>,
  "name": <clear and concise name for higher level cluster idea>
}
```
\end{Verbatim}

\hrule
\bigskip

\textbf{User Prompt}

\begin{Verbatim}[breaklines, fontsize=\fontsize{5}{6}\selectfont]
### Clusters

- {cluster name}: {cluster description}
- {cluster name}: {cluster description}
...
\end{Verbatim}

\end{flushleft}
\end{framed}
\caption{Prompt to create super cluster labels for groups of base clusters.} 
\label{fig:super_cluster_prompt}
\end{figure*}


\begin{figure*}
\begin{framed}
\begin{flushleft}
\noindent
\textbf{Deduplicate Super Clusters} \\

\medskip
\hrule
\bigskip

\textbf{System Prompt}

\begin{Verbatim}[breaklines, fontsize=\fontsize{5}{6}\selectfont]
You are tasked with deduplicating a list of cluster names and descriptions into a smaller set of distinct clusters.
Your goal is to create relatively distinct clusters that can best represent the original list.

## Context
The user will provide a list of clusters including their names and descriptions.
This cluster list will be used to categorize diverse data points.
You should ensure that to deduplicate the list to only retain distinctive clusters that do not overlap with each other.

## Instruction
  1. Analyze the given list of cluster names to identify similarities, patterns, and themes.
  2. Group similar cluster names together based on their semantic meaning, not just lexical similarity.
  3. For each group, select a representative name that best captures the essence of the cluster. This can be one of the original clusters' name or a new name that summarizes the group effectively.
  4. Merge the most similar groups until you reach the desired number of clusters. Maintain as much specificity as possible while merging.
  5. You should write a representative description for the new cluster. Maintain the specificity of original clusters' description.
  6. Ensure that the final set of cluster names are distinct from each other and collectively represent the diversity of original list.
  7. Avoid significantly reducing the original list. The user will provide a target length for the new list.
  8. You do not have to modify or re-create all of the cluster. You should **modify them only when you feel it is necessary**. If not, you can just leave the cluster as is.
  9. Ensure that you use the same language as the original clusters for the new cluster names and descriptions.

## Response Format (in JSON)
```json
{
  "justification": <your detailed explanation about the final answer according to instruction>,
  "finals": [
    {
      "name": <new cluster name>,
      "description": <new cluster description>
    },
    ... <new clusters> ...
  ]
}
```
\end{Verbatim}

\hrule
\bigskip

\textbf{User Prompt}

\begin{Verbatim}[breaklines, fontsize=\fontsize{5}{6}\selectfont]
### Clusters

- {cluster name}: {cluster description}
- {cluster name}: {cluster description}
...
\end{Verbatim}

\end{flushleft}
\end{framed}
\caption{Prompt to deduplicate similar super clusters.} 
\label{fig:super_cluster_deduplication_prompt}
\end{figure*}


\begin{figure*}
\begin{framed}
\begin{flushleft}
\noindent
\textbf{Base Cluster-Super Cluster Reassignment} \\

\medskip
\hrule
\bigskip

\textbf{System Prompt}

\begin{Verbatim}[breaklines, fontsize=\fontsize{5}{6}\selectfont]
You are tasked with categorizing a specific cluster into one of the provided higher-level clusters based on their relevance and similarity.
Your goal is to determine which higher-level cluster best fits the given specific cluster based on its name and description.

## Context
The user will provide the name and description of one lower level cluster and a list of higher level clusters.
You should categorize the lower level cluster into the most relevant higher level cluster.

## Instruction
1. Analyze the name and description of the lower level cluster.
2. Consider the key characteristics, themes, or subject matter of the lower level cluster.
3. Compare these elements to the higher level clusters provided.
4. Determine which higher level cluster best encompasses the lower level cluster. You MUST assign the lower cluster to the most  suitable higher level cluster, even if multiple higher level clusters are relevant.
5. Make sure you pick the most sensible cluster based on the information provided.

## Response Format (in JSON)
```json
{
  "justification": <Justify why you assign the lower level cluster to the answer higher level cluster>,
  "cluster": <the index number of higher level cluster>
}
```
\end{Verbatim}

\hrule
\bigskip

\textbf{User Prompt}

\begin{Verbatim}[breaklines, fontsize=\fontsize{5}{6}\selectfont]
### Target Cluster

- {cluster name}: {cluster description}

### Higher Cluster
- {cluster name}: {cluster description}
- {cluster name}: {cluster description}
...
\end{Verbatim}

\end{flushleft}
\end{framed}
\caption{Prompt to reassign base clusters to more relevant super clusters.} 
\label{fig:super_cluster_reassignment_prompt}
\end{figure*}

