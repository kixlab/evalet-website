\begin{figure*}
\begin{framed}
\begin{flushleft}
\noindent
\textbf{Deduplicate Super Clusters} \\

\medskip
\hrule
\bigskip

\textbf{System Prompt}

\begin{Verbatim}[breaklines, fontsize=\fontsize{5}{6}\selectfont]
You are tasked with deduplicating a list of cluster names and descriptions into a smaller set of distinct clusters.
Your goal is to create relatively distinct clusters that can best represent the original list.

## Context
The user will provide a list of clusters including their names and descriptions.
This cluster list will be used to categorize diverse data points.
You should ensure that to deduplicate the list to only retain distinctive clusters that do not overlap with each other.

## Instruction
  1. Analyze the given list of cluster names to identify similarities, patterns, and themes.
  2. Group similar cluster names together based on their semantic meaning, not just lexical similarity.
  3. For each group, select a representative name that best captures the essence of the cluster. This can be one of the original clusters' name or a new name that summarizes the group effectively.
  4. Merge the most similar groups until you reach the desired number of clusters. Maintain as much specificity as possible while merging.
  5. You should write a representative description for the new cluster. Maintain the specificity of original clusters' description.
  6. Ensure that the final set of cluster names are distinct from each other and collectively represent the diversity of original list.
  7. Avoid significantly reducing the original list. The user will provide a target length for the new list.
  8. You do not have to modify or re-create all of the cluster. You should **modify them only when you feel it is necessary**. If not, you can just leave the cluster as is.
  9. Ensure that you use the same language as the original clusters for the new cluster names and descriptions.

## Response Format (in JSON)
```json
{
  "justification": <your detailed explanation about the final answer according to instruction>,
  "finals": [
    {
      "name": <new cluster name>,
      "description": <new cluster description>
    },
    ... <new clusters> ...
  ]
}
```
\end{Verbatim}

\hrule
\bigskip

\textbf{User Prompt}

\begin{Verbatim}[breaklines, fontsize=\fontsize{5}{6}\selectfont]
### Clusters

- {cluster name}: {cluster description}
- {cluster name}: {cluster description}
...
\end{Verbatim}

\end{flushleft}
\end{framed}
\caption{Prompt to deduplicate similar super clusters.} 
\label{fig:super_cluster_deduplication_prompt}
\end{figure*}

