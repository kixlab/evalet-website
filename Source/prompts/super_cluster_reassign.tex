\begin{figure*}
\begin{framed}
\begin{flushleft}
\noindent
\textbf{Base Cluster-Super Cluster Reassignment} \\

\medskip
\hrule
\bigskip

\textbf{System Prompt}

\begin{Verbatim}[breaklines, fontsize=\fontsize{5}{6}\selectfont]
You are tasked with categorizing a specific cluster into one of the provided higher-level clusters based on their relevance and similarity.
Your goal is to determine which higher-level cluster best fits the given specific cluster based on its name and description.

## Context
The user will provide the name and description of one lower level cluster and a list of higher level clusters.
You should categorize the lower level cluster into the most relevant higher level cluster.

## Instruction
1. Analyze the name and description of the lower level cluster.
2. Consider the key characteristics, themes, or subject matter of the lower level cluster.
3. Compare these elements to the higher level clusters provided.
4. Determine which higher level cluster best encompasses the lower level cluster. You MUST assign the lower cluster to the most  suitable higher level cluster, even if multiple higher level clusters are relevant.
5. Make sure you pick the most sensible cluster based on the information provided.

## Response Format (in JSON)
```json
{
  "justification": <Justify why you assign the lower level cluster to the answer higher level cluster>,
  "cluster": <the index number of higher level cluster>
}
```
\end{Verbatim}

\hrule
\bigskip

\textbf{User Prompt}

\begin{Verbatim}[breaklines, fontsize=\fontsize{5}{6}\selectfont]
### Target Cluster

- {cluster name}: {cluster description}

### Higher Cluster
- {cluster name}: {cluster description}
- {cluster name}: {cluster description}
...
\end{Verbatim}

\end{flushleft}
\end{framed}
\caption{Prompt to reassign base clusters to more relevant super clusters.} 
\label{fig:super_cluster_reassignment_prompt}
\end{figure*}

